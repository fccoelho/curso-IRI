\documentclass[compress]{beamer}
\usetheme{Warsaw}
\usecolortheme{crane}
\usepackage[utf8]{inputenc}
\usepackage[portuguese]{babel}
\usepackage{url}
\usepackage{tikz}
\usetikzlibrary{arrows}
\usetikzlibrary{arrows}
\usepackage{default}
\usepackage{amsmath}
\usepackage{amsfonts}
\usepackage{amssymb}

% Algumas definições %%%
\def\term#1{{\sc #1}}   % IR terms in examples not index terms!
\def\query#1{{\sf #1}}
\def\oper#1{{\sc #1}} % AND, OR, NOT
%%%%%%%%%%%%%%%%%%%%%%%%

\title[Coelho: Vocabulário de termos e lista de ``postings'']
{Introdução à Recuperação de Informações\\
\large \url{https://github.com/fccoelho/curso-IRI}\\[0.5cm]
IRI 2: Vocabulário de termos e lista de ``postings''}

\author [Coelho F.C. \& Souza R.R.]{ Flávio Codeço Coelho}

\institute [EMAp, FGV]{Escola de Matemática Aplicada,   Fundação Getúlio Vargas}
\date


\begin{document}

\begin{frame}
\titlepage
\end{frame}

\begin{frame}[fragile]
\frametitle{Sumário da Aula}
\tableofcontents
\end{frame}
\section{Recapitulação}
\begin{frame}
\frametitle{Índice invertido}
Para cada termo $t$, armazenamos uma lista de todos os documentos que contém 
$t$.

\bigskip
\begin{tabular}{|c|c|r|r|r|r|r|r|r|r|r|}
\cline{1-1}\cline{3-10}
\term{Brutus} & $\longrightarrow$ & 1 & 2 & 4 & 11 & 31 & 45 & 173 & 174 \\ 
\cline{1-1}\cline{3-10}
\multicolumn{8}{l}{} \\ \cline{1-1}\cline{3-11}
\term{Caesar} & $\longrightarrow$ & 1 & 2 & 4 & 5 & 6 & 16 & 57 & 132 & \ldots 
\\ \cline{1-1}\cline{3-11}
\multicolumn{8}{l}{} \\ \cline{1-1}\cline{3-6}
\term{Calpurnia} & $\longrightarrow$ & 2 & 31 & 54 & 101 \\
\cline{1-1}\cline{3-6} \multicolumn{8}{l}{}  \\
% The multicolumn{1}'s below suppress the vertical lines....
\multicolumn{1}{c}{$\vdots$} \\
\multicolumn{1}{c}{$\underbrace{\phantom{\mbox{Calpurnia}}}$} &
\multicolumn{1}{c}{} &
\multicolumn{9}{c}{$\underbrace{\phantom{\mbox{Calpurnia Calpurnia
Calpurnia Caesar hath}}}$} \\
\multicolumn{1}{c}{\visible<1>{\textbf{dictionary}}} &
\multicolumn{1}{c}{} & \multicolumn{9}{c}{\visible<1>{\textbf{postings}}}
\end{tabular}
\end{frame}

\begin{frame}
\frametitle{Interseção de duas listas de ``postings''}

\begin{tabular}{lll}
\term{Brutus} & $\longrightarrow$ &
\alert<2>{\framebox{1}}$\rightarrow$\alert<3,4>{\framebox{2}}
$\rightarrow$\alert<5>{\framebox{4}}$\rightarrow$\alert<6>{\framebox{11}}
$\rightarrow$\alert<7,8>{\framebox{31}}$\rightarrow$\alert<9>{\framebox{45}}
$\rightarrow$\alert<10,11>{\framebox{173}}$\rightarrow$\framebox{174}\\[1ex]
\term{Calpurnia} & $\longrightarrow$ &
\alert<2-4>{\framebox{2}}$\rightarrow$\alert<5-8>{\framebox{31}}
$\rightarrow$\alert<9-10>{\framebox{54}}$\rightarrow$\alert<11>{\framebox{101}}
\\[2ex]
Intersection & $\Longrightarrow$ & 
\visible<4->{\framebox{2}}\visible<8->{$\rightarrow$\framebox{31}}
\end{tabular}

\end{frame}

\begin{frame}
\frametitle{Construindo o índice invertido: ordenando os ``postings''}
\begin{tiny}
\mbox{
\begin{tabular}{@{}lr}
\textbf{term} & \textbf{\llap{doc}ID} \\
I & 1 \\
did & 1 \\
enact & 1 \\
julius & 1 \\
caesar & 1 \\
I & 1 \\
was & 1 \\
killed & 1 \\
i' & 1 \\
the & 1 \\
capitol & 1 \\
brutus & 1 \\
killed & 1 \\
me & 1 \\
so & 2 \\
let & 2 \\
it & 2 \\
be & 2 \\
with & 2 \\
caesar & 2 \\
the & 2 \\
noble & 2 \\
brutus & 2 \\
hath & 2 \\
told & 2 \\
you & 2 \\
caesar & 2 \\
was & 2 \\
ambitious & 2
\end{tabular}
%
{\huge $\Longrightarrow$}
%
\begin{tabular}{lr}
\textbf{term} & \textbf{\llap{doc}ID} \\
ambitious & 2 \\
be & 2 \\
brutus & 1 \\
brutus & 2 \\
capitol & 1 \\
caesar & 1 \\
caesar & 2 \\
caesar & 2 \\
did & 1 \\
enact & 1 \\
hath & 1 \\
I & 1 \\
I & 1 \\
i' & 1 \\
it & 2 \\
julius & 1 \\
killed & 1 \\
killed & 1 \\
let & 2 \\
me & 1 \\
noble & 2 \\
so & 2 \\
the & 1 \\
the & 2 \\
told & 2 \\
you & 2 \\
was & 1 \\
was & 2 \\
with & 2
\end{tabular}
}
\end{tiny}
\end{frame}

\begin{frame}
\frametitle{O Google usa busca Booleana?}
\begin{itemize}
\item No Google, a interpretação defaul de uma consulta 
  [$w_1$ $w_2$ 
\ldots $w_n$] é 
  $w_1$ AND $w_2$ AND 
\ldots AND $w_n$
\item Casos em que você recebe resultados que não contém uma das $w_i$:
\begin{itemize}
\item âncora de texto
\item página contém variantes da $w_i$ (morfologia, correção ortográfica
 , sinônimo)
\item consultas longas ($n$ é grande)
\item expressões booleanas geram poucos resultados.
\end{itemize}

\item Booleana simples vs.\ Ordenação do conjunto de resultados

\begin{itemize}
\item A busca booleana simples não gera ordenação dos documentos.
\item O Google (e a maioria dos buscadores booleanos bem feitos) ordena o 
conjunto de resultados -- com os melhores resultados (de acordo com um 
estimador de relevância) no topo.

\end{itemize}
\end{itemize}
\end{frame}

\section{Documentos}
\begin{frame}
\frametitle{Documentos}
\begin{itemize}[<+->]
\item Última aula: Sistema de recuperação Booleana simples
\item Nossos pressupostos eram:
\begin{itemize}[<+->]
\item Nós sabemos o que é um documento.
\item Documentos são ``legíveis por máquina''.
\end{itemize}
\item Na prática isto pode ser bem complicado.
\end{itemize}
\end{frame}

\begin{frame}
\frametitle{Parseando um Documento}
\begin{itemize}[<+->]
\item Precisamos lidar com formato e lingua de cada documento.
\item Em que formato ele está? pdf, word, excel, html etc.
\item Em que língua ele está?
\item Em que codificação está?
\item Cada uma destas perguntas implica um problema de 
classificação.
\item Alternativa: usar heurísticas.
\end{itemize}
\end{frame}

\end{document}
