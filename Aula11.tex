 
\documentclass[compress]{beamer}
\usetheme{Warsaw}
\usecolortheme{crane}
%%%Xelatex section
\usepackage{ifxetex}
\ifxetex
\usepackage{xltxtra}
\usepackage{polyglossia}
\setdefaultlanguage[⟨options⟩]{brazil}
\usepackage{fontspec,lipsum}
\defaultfontfeatures{Ligatures=TeX}
%\setromanfont{Georgia}
%\setsansfont{Tahoma}
%%%%
\else
\usepackage[utf8]{inputenc}
\usepackage[portuguese]{babel}
\fi
\usepackage{url}
\usepackage{default}
\usepackage{xcolor}
\usepackage{newalg}
\usepackage{pstricks,pst-node,pst-plot,pst-tree}

\definecolor{texblue}{rgb}{0, 0, 1} 
\def\myblue#1{\textcolor{texblue}{#1}}
\definecolor{texred}{rgb}{1, 0, 0} 
\def\myred#1{\textcolor{texred}{#1}}
\definecolor{texgreen}{rgb}{0, 1, 0} 
\definecolor{texgreen}{rgb}{0.2,0.6,0.2}
\def\mygreen#1{\textcolor{texgreen}{#1}}
\def\term#1{{\sc #1}}   % IR terms in examples not index terms!
\def\query#1{{\sf #1}}
\def\oper#1{{\sc #1}} % AND, OR, NOT
%%% Hinrich pstricks stuff

\newcommand{\Xeasquare}[3][]{\fnode[#1]{#2}{#3}}
\newcommand{\Xeanode}[3][]{\circlenode[#1]{#2}{#3}}
\newcommand{\XeaFnode}[2]{\circlenode[doubleline=true]{#1}{#2}}
\newcommand{\Xeatrans}[4][]{\ncline[#1]{->}{#2}{#3}\mput*{#4}}
\newcommand{\Xeaarc}[4][8]{\ncarc[arcangleA=#1, arcangleB=#1]{->}{#2}{#3}\mput*{#4}}
\newcommand{\Xeacurve}[4][]{\nccurve[#1]{->}{#2}{#3}\mput*{#4}}
\newcommand{\Xealoop}[3][angleA=-30, angleB=30]{\nccurve[ncurv=4, #1]{->}{#2}{#2}\Bput*{#3}}
\newcommand{\Xeastart}[1]{\ncdiag[angleA=180, angleB=180, arm=.5, arrowsize=4pt 4]{->}{#1}{#1}}
\newcommand{\eaStateName}{d}
\newcommand{\eanode}[2][]{\Xeanode[#1]{q#2}{${\eaStateName}_{#2}$}}
\newcommand{\easquare}[2][]{\Xeasquare[#1]{q#2}{${\eaStateName}_{#2}$}}
\newcommand{\eaFnode}[1]{\XeaFnode{q#1}{${\eaStateName}_{#1}$}}
\newcommand{\eatrans}[4][]{\Xeatrans[#1]{q#2}{q#3}{#4}}
\newcommand{\eaarc}[4][8]{\Xeaarc[#1]{q#2}{q#3}{#4}}
\newcommand{\eacurve}[4][]{\Xeacurve[#1]{q#2}{q#3}{#4}}
\newcommand{\ealoop}[3][angleA=-30, angleB=30]{\Xealoop[#1]{q#2}{#3}}
\newcommand{\eastart}[1]{\Xeastart{q#1}}
\long\def\eateat#1{\ignorespaces}


\title[Recuperação Probabilística\hspace{2em}\insertframenumber/\inserttotalframenumber]
{Sistemas de Recuperação de Informação\\
\large \url{https://github.com/fccoelho/curso-IRI}\\[0.5cm]
IRI 11: Recuperação de Informação Probabilística}

\author [Coelho F.C. \& Souza R.R.]{ Flávio Codeço Coelho}

\institute [EMAp, FGV]{Escola de Matemática Aplicada,   Fundação Getúlio Vargas}
\date


\begin{document}

\begin{frame}
\titlepage
\end{frame}

\begin{frame}[fragile]
\frametitle{Sumário da Aula}
\tableofcontents
\end{frame}

\section{Recapitulação}

\begin{frame}
\frametitle{Revisão de relevância: Ideia básica}

\begin{itemize}
\item O usuário faz uma consulta simples, curta.
\item O buscador retorna um conjunto de documentos.
\item O usuário marca alguns documentos como relevantes outros não.
\item Buscador computa nova representação da informação requerida -- deve ser melhor que a consulta inicial.
\item Buscador executa nova consulta e retorna resultados.
\item Novos resultados apresentação melhor revocação (espera-se).
\end{itemize}
\end{frame}

\frame{
\frametitle{Rocchio}

\psset{unit=0.175cm}


\begin{pspicture}(5,5)(45,36)
\psaxes[labels=none,ticks=none,arrowscale=3,linewidth=0.02cm]{->}(15,10)(45,36)

%\showgrid
%\psgrid[subgriddiv=0](0,2)(55,36)

\pscircle( 30 ,11  ){0.428}
\pscircle( 30 ,13  ){0.428}
\pscircle( 30 ,15  ){0.428}
\pscircle( 30 ,17  ){0.428}
\pscircle( 30 ,23  ){0.428}
\pscircle( 30 ,25  ){0.428}
\pscircle( 30 ,27  ){0.428}
\pscircle( 30 ,29  ){0.428}
\pscircle( 33 ,20  ){0.428}
\pscircle( 27 ,20  ){0.428}

\rput( 40 ,25  ){{ x}}
\rput( 40 ,15  ){{ x}}
\rput( 40 ,23  ){{ x}}
\rput( 40 ,17  ){{ x}}
\rput( 43 ,20  ){{ x}}
\rput( 37 ,20  ){{ x}}

%\pscircle*(40,20){0.711}
%\pscircle*(30,20){0.711}

\visible<2,3,5-9>{
\psline{->,arrowscale=2,linewidth=0.03cm}(15,10)(30,20)
\rput(21,16){\small $\vec{\mu}_{R}$}
}

\visible<4,5-9>{
\psline{->,arrowscale=2,linewidth=0.03cm}(15,10)(40,20)
\rput(25,12){\small $\vec{\mu}_{NR}$}
}

\visible<6-9>{
\psline{->,arrowscale=2,linewidth=0.03cm}(40,20)(30,20)
\rput(35,23){\small $\vec{\mu}_{R}-\vec{\mu}_{NR}$}
}
\visible<7-9>{\psline{->,arrowscale=2,linewidth=0.03cm}(30,20)(20,20)}
\visible<8-10>{
\psline{->,arrowscale=2,linewidth=0.03cm}(15,10)(20,20)
\rput(20,22){\small $\vec{q}_{opt}$}
}

\end{pspicture}

}


\begin{frame}
\frametitle{Tipos de expansão de consulta}
%\pause[2]
\begin{itemize}
\item Tesauro manual (mantido por editores, p.ex., PubMed)
\item Tesauro derivado automaticamente (p.ex., baseado em estatísticas de co-ocurrence statistics)
\item Query-equivalence based on query log mining (common on
  the web as in the ``palm'' example)
\end{itemize}
\end{frame}

\begin{frame}
\frametitle{Expansão de Consulta em Buscadores}
%\pause[2]
\begin{itemize}
\item Fonte principal de expansões de consulta em buscadores: logs de consulta
\item Exemplo 1: Depois de consultar por [herbal], usuários frequentemente buscam por [remédio herbal].
\begin{itemize}
\item $\rightarrow$ 
``remédio herbal'' é uma expansão em potencial para ``herbal'' ou ``erva''.
\end{itemize}

\item Exemplo 2: Usuários buscando por [fotos de flores] frequentemente clicam na URL \myblue{photobucket.com/flor}. 
Usuários buscando por [desenhos de flor]
  frequentemente clicam na \myblue{mesma URL}.
\begin{itemize}
\item $\rightarrow$ 
``desenhos de flor'' e ``fotos de flor'' São potencialmente extensões uma da outra.
\end{itemize}

\end{itemize}
\end{frame}


\begin{frame}[Conclusão de Hoje]

\frametitle{Conclusão de Hoje}
\begin{itemize}
\item Abordagem probabilistica a RI
\item Principio de Rankeamento de probabilidade
\item Modelos: BIM, BM25
\item Pressupostos destes modelos
\end{itemize}
\end{frame}


\end{document}